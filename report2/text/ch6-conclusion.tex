\chapter{\label{ch:7-conclusion} Conclusion and Future Work}

%\minitoc


\section{Review of Aims and Objectives}
I would say that this project has been moderately successful. Some feasible movie review text has indeed been generated, although not a lot and not in quite as large quantity or variety as I had aimed.
\subsection{Aims}
The aim of this project was to generate movie review prose that passed as human. This has been partially achieved, at the sentence level, although a movie review generated by any of my systems is not likely to pass as human, mostly due to the falling-apart of coherence outside of reporting facts and sentiments. I would not say this was achieved with great success, as little interaction with any of the systems aiming to pass as human has occurred.
\subsection{Objectives}
To explore methods for generating text which reviews film.\\
This object has certainly been met. I have explored a number of options for generating prose and implemented several methods for generating prose. I have looked at Markov models, Feed-Forward Neural Networks, Template based systems using context-free grammars, and Natural Language Generation architectures for generating prose.\\

To develop a natural language generation system which picks points from multiple review texts and uses them to structure an informed review.\\
Although I have implemented a system to meet his goal, the outputs generated do not manage to pass as human, and therefore I would not say I have met this objective with great success. As well as this, I made the concession of including templates into the NLG system in an attempt at making the system more human-like. More time to work on this system and I feel I could have met this objective more, although it would be partially template-based rather than strictly NLG.\\
To develop an online platform to host generated movie reviews which can gather feedback and data on the reception of said review.\\
I have met this objective successfully, although it had not been used as much as I had hoped for the collection of data. I could have put more time into the development of the site in order to make it more welcoming as well as hosting it on a URL that is potentially less scary than the URL of my system. (doc.gold.ac.uk/~tpalm003/reviews/index.php).\\
To develop an autonomous bot that can discuss movies (or at least reply with an opinion of a movie) over twitter. Ideally this would drive traffic towards the movie review blog which hosts reviews made by the bot.\\
This goal has been met, although its outreach is limited by the behaviours it puts into place. It makes posts and can post with hash-tags in an attempt to reach an audience, but it does not engage with twitter users through other interactions which would arguably make it a more believable bot, and at the very least have it seen by more people.\\


\section{Lessons Learned}
I have learned a number of things from this project, and had I started it again have a number of things I would do differently.
At the programming level, I've learnt a lot about using APIs as well as interracting with XML and JSON objects, as well as parsing text and working around very error-prone systems such as requesting data that might not necessarily exist.\\
I have had to research language processing and sentiment tagging, although I did opt to use libraries for these in my implementation as there was little point reimplementing code for systems with so many interacting elements.\\
I'd also not recommend relying on user data for systems that require finding attention in the real world (rather than appealing to users for feedback), as this has proven to be difficult to manage. As well as this I would recommend that people choose less subjective topics to implement NLG systems for if they were to choose NLG as a dissertation topic.


\section{Future Work}


\subsection{Improvements to current work}
An interesting area to explore is expanding the Twitter bot to handle behaviours other than self promotion and Markov chain generated text to reply to others. This would enable a system which intended to draw more attention to itself to operate in a more human-like manner as a real twitter user is more likely to utilize all of the features of the website and have a greater outreach through the use of replies - as notifications are sent directly to the user rather than having to search for a tweet generated by the bot.\\


It would also be interesting to expand the NLG system to handle a larger range of topics within movie generation for a more believable and comprehensive review of the film, and expand the grammars used to generate the sentiment-driven sentences. This would mostly include much more specific sentiment analysis using lists of sentiment tagged words for specific topics and themes that occur in cinema.\\

A further interesting expansion would be to target reviews at different audiences where people actively read for them, such as Youtube comments, reviews for Amazon streaming videos, and other websites that host users movie watchlists and reivews such as IMdB and letterboxd, as they have active communities that use will read these reviews to gage whether or not they want to watch a film, or for other purposes such as validating their own opinions. 

\section{The Future of Generative Film Reviews}
It is definitely safe to assume that the human-written review will not be made obsolete by reviews created from NLG methods for a long while yet. Given that a system like this would need to build its knowledge from some agent with enough insight to identify talking points and provide sentiment for each of these talking points, it is safe to say that without brilliant leaps in computer vision and audio processing, an article penned by a trusted reviewer is not going to be made redundant any time soon.

