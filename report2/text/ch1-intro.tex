%\begin{savequote}[8cm]
%\textlatin{Neque porro quisquam est qui dolorem ipsum quia dolor sit amet, consectetur, adipisci velit...}
%
%There is no one who loves pain itself, who seeks after it and wants to have it, simply because it is pain...
%  \qauthor{--- Cicero's \textit{de Finibus Bonorum et Malorum}}
%\end{savequote}

\chapter{\label{ch:1-intro}Introduction} 

\minitoc

\section{Overview of the System}

There are several systems I have implemented in this project. The first of which is a twitter bot designed to discuss film and drive traffic towards the NLG system's posts, which are hosted on a blog I have implemented in PHP with a mySQL database. It features a comments section and a feedback forum for collecting information on how effective the posts generated are.\\
There are a few systems used to create reviews. A Markov Chain based model which creates review text based on a corpora and probabilistically determines what is to be posted.
The second, a template driven system which Choses sentences from a template and attempts to fill them. 


\section{Motivation}

The Internet is a vast resource for opinion, thoughts and discourse on many topics such as film and other media. There are countless reviews, ratings and comments about any given topic, and this is a valuable resource to mine in order to extract opinion and detail about what is being discussed.\\

The applications of Natural Language Processing (NLP) and more specifically Natural Language Generation (NLG) are powerful in this domain.  Opinion mining and understanding of such a vast field of reviewers and people engaging in discussion can provide interesting data and context on the success of a film. Such methods are able to process and understand a very large corpus of text far faster than one may be able to read through all of the writings on a film manually. \\

An issue with summarisation and opinion mining of corpora such as these is that they do not necessarily provide representative criticisms or feedback on a film in the user-facing output, often only chunks of text that are deemed the most representative and metrics that are as simple as a positive or negative rating, or a list of keywords that have been extracted.\\
%come back to this?

This project aims to create a system that solves these issues, and generates a review of a movie that is both coherent and insightful, related to the corpus of movie review text it is given.\\

A system of this kind could be employed in business - for example, reviews and articles about cinema can have a profound effect on their commercial success, and if enough respected reviewers pan a film it may become necessary to understand why - and a tool such as this could aid this process.\\

It could also be employed at a consumer level in order for a user to quickly evaluate whether or not they wanted to watch a film or buy a product based on a vast amount of review text that exists rather than the opinion of a singular reviewer.

\section{Contribution}


\section{Thesis Structure}
This thesis is structured as follows: In Chapter~\ref{ch:2-litreview} a detailed background research in the  area of \emph{Fraud Detection} is given. In Chapter~\ref{ch:3-design} the system design part is describe. 
 In chapter~\ref{ch:4-implementation} the implementation part of the system is given. Chapter~\ref{ch:5-testing} 
 consists of description of the testing techniques an the testing results. Finally, a conclusion and a direction of future work is given in  Chapter~\ref{ch:6-conclusion}. 